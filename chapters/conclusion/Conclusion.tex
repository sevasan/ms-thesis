% !TEX root = ../../main.tex

В этой работе мы исследовали однопетлевое эффективное действие теории Янга--Миллса ($\ref{eq:EffAction1}$) с помощью метода теплового ядра ($\ref{eq:LogDetHeatKernel}$). Для получения конечных (в инфракрасном пределе) вкладов в эффективное действие, предлагается представление для следа теплового ядра в виде интеграла по путям ($\ref{eq:HeatKernelFinal}$) (см. разделы \ref{sec:HeatKernelDiscussion} и \ref{sec:DerivExpanResult}, а также \cite{Mnev}).

В качестве примера применения формулы ($\ref{eq:HeatKernelFinal}$), был представлен способ получения градиентного разложения эффективного действия ($\ref{eq:EffLagrangianDerivExpansion}$) по производным тензора напряжённости $F_{\mu \nu}$ (поле медленно меняющейся амплитуды) для калибровочной группы $SU(2)$. Низший порядок этого разложения ($\ref{eq:EffLagrangianCovarConst}$) совпадает с известным результатом для ковариантно--постоянного поля \cite{Savvidy1977} (полученным другим способом). Это вычисление~--- простейший пример разложений, которые можно получить из формулы ($\ref{eq:HeatKernelFinal}$). Отметим, что результат ($\ref{eq:EffLagrangianDerivExpansion}$) хорошо определён в инфракрасной области~--- его можно интегрировать по собственному времени $t$.

Наши результаты теперь можно применить к исследованию вакуумных глюонных возбуждений. А именно доказать существование неоднородных, т.н. <<солитонных>> конфигураций поля, ведь  в квантовой теории Янга--Миллса появляется размерный параметр, который мог бы характеризовать массу (размер) солитона. Эта идея принадлежит Фаддееву \cite{Faddeev2008}. В его работе с Ниеми \cite{Faddeev2007} рассматривается полная замена переменных для калибровочного поля $SU(2)$, которая должна упрощать описание теории в инфракрасной области. Согласно их программе, замену переменных необходимо проводить уже в эффективном действии ($\ref{eq:EffAction1}$). Возможным способом реализации этой идеи является использование представления ($\ref{eq:HeatKernelFinal}$) для теплового ядра и аналога градиентного разложения ($\ref{eq:EffLagrangianDerivExpansion}$), но уже относительно новых переменных Фаддеева--Ниеми. Если при этом удастся доказать возникновение ненулевого вакуумного среднего $\langle \rho^2\rangle$, то это объяснит механизм возникновения глюонного конденсата и будет очень важным шагом на пути объяснения конфайнмента.