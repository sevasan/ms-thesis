% !TEX root = ../../main.tex

Итак, калибровочно--инвариантный результат для $\Gamma_1^{-\nabla^2}$ (в случае фундаментального представления $J=\frac12$) даётся формулой ($\ref{eq:Gamma1Answer}$).
Итоговое выражение для градиентного разложения однопетлевого эффективного действия:
\begin{equation}
	\label{eq:EffLagrangianDerivExpansion}
	\mathcal{L}^1=-\frac{1}{16\pi^2}\int\limits_0^\infty\!\frac{\D t}{t^2}\frac{F}{\sin{Ft}}\frac{\left(e^{-iFt}+e^{iFt}\right)}{2}\left(1+\sum\limits_{i=1}^3\left[\frac12 \tr\,\Gamma_i^{-\nabla^2-2iF_{\mu \nu}}-\Gamma_i^{-\nabla^2}\right]+\dots\right),
\end{equation}
где множитель перед скобкой~--- результат для коваринатно--постоянного поля, а $\Gamma_i^M$~--- петлевая поправка для соответствующего оператора $M=-\nabla^2,\,-\nabla^2-2iF_{\mu \nu}$ ($\ref{eq:Gamma1}$--$\ref{eq:Gamma3}$) и $\tr$ подразумевается  по лоренцевским значкам. Здесь надо иметь в виду, что интегрирование по калибровочной группе (см. раздел \ref{sec:SU2Integration}) изменит пропагатор $G_{\mu \nu}$ (см. раздел \ref{sec:GreensFinction}). Видимо, при пертурбативном учёте поправок от производных $F_{\mu \nu}$, их вклад в ($\ref{eq:EffLagrangianDerivExpansion}$) будет степенным при $t\to\infty$. При этом расходимость (которая даёт мнимую часть в эффективном действии ($\ref{eq:EffLagrangianCovarConst}$)) носит экспоненциальный характер. Поэтому имеет смысл подставлять в ($\ref{eq:HeatKernelPathIntegral}$) более сложные полевые конфигурации, которые можно получить с помощью разложения Фаддеева--Ниеми \cite{Faddeev2007}. Проблемой в этом подходе будет взять интеграл по $SU(2)$, аналогичный ковариантно--постоянному случаю (см. раздел $\ref{sec:CovarConstField}$). Вероятно, здесь можно будет продвинуться, используя приближение $\rho=\text{Const}$ (см. \cite{Faddeev2007}).