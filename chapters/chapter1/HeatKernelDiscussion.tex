% !TEX root = ../../main.tex

Разложение Сили--деВитта ($\ref{eq:SilleyWitt}$) подходит, в основном, для исследования ультрафиолетового поведения $t\to 0$ интегралов в ($\ref{eq:LogDetHeatKernel}$). В частности, относительно легко можно получить $\beta$--функцию теории Янга--Миллса (в согласии с \cite{GrossWilczek1973,Politzer1973}). Однако, эта формула не подходит для интегрирования по собственному времени в ($\ref{eq:LogDetHeatKernel}$), так как каждое слагаемое будет расходиться на верхнем (инфракрасном) пределе $t\to \infty$. Поэтому для изучения конечных поправок в однопетлевом эффективном действии ($\ref{eq:EffAction1}$) необходимо использовать представление ($\ref{eq:HeatKernelPathIntegral}$). Объединяя этот результат с формулой Дьяконова--Петрова ($\ref{eq:DPFormulaGeneral}$), получим:
\begin{equation}
	\label{eq:HeatKernelFinal}
	K_M(t)=\mkern-\thickmuskip\int\mkern-\thickmuskip\mathcal{D}z\mathcal{D}U \exp{\left\{\int\limits_0^t\!\D \tau\!\left( -\frac{\dot{z}^2}{4} + iJ\tr\,\sigma_3\left(U^\dagger A_\mu U\dot{z}_\mu + iU^\dagger BU+iU^\dagger\dot{U}\right) \right)\right\}},
\end{equation}
где на переменные интегрирования наложены периодические граничные условия: $z(0)=z(t)$ и $U(0)=U(t)$, при этом $z(\tau)\in \mathbb{R}^4$, а $U(\tau)\in SU(2)$, а $M=-\nabla^2+B$.

Формула ($\ref{eq:HeatKernelFinal}$) может быть отправной точкой для нахождения с конечных поправок в эффективном действии. При этом на поле $A_\mu$ необходимо накладывать некоторые условия, которые дадут возможность продвинуться в вычислении интеграла ($\ref{eq:HeatKernelFinal}$). Например, можно считать поле $A_\mu$ ковариантно--постоянным и получить уже известный результат \cite{Savvidy1977} (см. также \cite{Mnev}). Однако, в этом случае вакуум, как уже отмечалось, будет неустойчивым. Альтернативой этому пути может быть использование переменных Фаддеева--Ниеми \cite{Faddeev2007}, с помощью которых можно рассматривать различные нетривиальные полевые конфигурации. В частности, интересно было бы написать разложение по производным плотности глюонного конденсата $\rho$ (введённой в работе \cite{Faddeev2007}). Но для этого сначала надо понять, как строится более простое разложение по производным тензора напряженности $F_{\mu \nu}$ (в случае $SU(2)$), так как это будет обобщением результата для ковариантно--постоянного поля и даст основу для дальнейших вычислений в более сложных анзацах.