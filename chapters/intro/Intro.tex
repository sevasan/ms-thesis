% !TEX root = ../../main.tex

Теория Янга--Миллса -- это калибровочная теория поля, лежащая в основе Стандартной Модели элементарных частиц. Она обобщает квантовую электродинамику на случай неабелевой калибровочной группы. С её помощью строится квантовая хромодинамика (КХД)~--- теория сильных взаимодействий кварков и глюонов. Одним из главных открытых вопросов в КХД является описание квантовых эффектов в пределе низких энергий (инфракрасный режим). Известно, что в этой области существует конфайнмент~--- невозможность получения свободных кварков, так как на эксперименте наблюдают только составные состояния из нескольких кварков. Для объяснения этого эффекта была предложена модель дуального сверхпроводника (см. например, обзор \cite{Ripka2003}), необходимым элементом которой является наличие конденсата глюонной плотности. Однако механизм его получения из исходной теории до сих пор неизвестен. В этой работе мы исследуем эффективное действие теории Янга--Миллса, с помощью которого можно получить информацию о структуре глюонных вакуумных возбуждений. Возможно при этом получится доказать наличие конденсата и, как следствие, объяснить конфайнмент.

Квантовые теории поля, как правило, описываются с помощью лагранжианов. Лагранжиан Янга--Миллса имеет следующий вид:
\begin{equation*}
	\mathcal{L}_{\text{YM}}[A]=-\frac{1}{4g^2}\, F^a_{\mu \nu}F^{a\mu \nu},
\end{equation*}
где $F^a_{\mu \nu}=\partial_\mu A^a_\nu - \partial_\nu A^a_\mu + f^{abc}A^b_\mu A^c_\nu$~--- тензор напряжённости, $f^{abc}$~--- структурные константы калибровочной группы, $d$~--- размерность пространства, $g$~--- константа связи. Отличительной чертой случая $d=4$ является тот факт, что константа связи $g$ здесь безразмерна.

Квантование теории Янга--Миллса удобно проводить в формализме функционального интеграла. В этой работе мы будет исследовать однопетлевое эффективное действие $\mathcal{S}^{\text{eff}}$~--- модифицированное действие теории, учитывающее квантовые поправки:
\begin{equation*}
	\exp{\left\{i\mathcal{S}^{\text{eff}}[A]\right\}}=\!\int\!\mathcal{D}a\,\exp{\left\{i\!\int\!\D^4 x\,\mathcal{L}_{\text{YM}}[A+a]\right\}}.
\end{equation*}
С помощью этого объекта можно получить информацию о корреляторах, вакуумных средних от операторов и т. д. Термин <<однопетлевое>> означает, что учитываются только низшие квантовые поправки по постоянной Планка $\hbar$.

Однопетлевое эффективное действие в теории Янга--Миллса представляет собой разность логарифмов детерминантов двух операторов:
\begin{align}
	\label{eq:EffAction1}
	\mathcal{S}^{1}=\frac 12 \ln\det\left((-\nabla^2)^{ab} - 2if^{acb}F^c_{\mu \nu}\right) - \ln\det\left(-\nabla^2\right),
\end{align}
где $\nabla_\mu=\partial_\mu-iA_\mu$~--- ковариантная производная. В этой формуле первый член соответствует вкладу калибровочного поля, второй~--- вкладу духов Фаддеева--Попова, которые вводятся в теорию в процессе фиксирования калибровки. Слагаемые в ($\ref{eq:EffAction1}$) сами по себе плохо определены и нуждаются в регуляризации. Именно из--за этого в квантовой теории Янга--Миллса появляется параметр, играющий роль массы (или размера), хотя классическая теория Янга--Миллса не содержит размерных констант связи и является масштабно--инвариантной. Это явление называется размерной трансмутацией. Для регуляризации логарифмов детерминантов в ($\ref{eq:EffAction1}$) обычно используют метод теплового ядра \cite{Faddeev2014} (или, что тоже самое, метод собственного времени Фока \cite{Fock}):
\begin{equation*}
	\mathcal{S}^1[A]=\int\limits_0^{\infty}\,\frac{\D t}{t}D(A,t),
\end{equation*}
где $D(A,t)$~--- тепловое ядро. Если интересоваться только ультрафиолетовой расходимостью $t\to 0$, то для теплового ядра можно написать разложение Сили--деВитта по степеням $t$ и получить такой результат:
\begin{equation*}
	D(A,t)=\frac{1}{g^2}\beta \int\limits_0^\infty \, \D^4 x\, \tr\,F_{\mu \nu}^2+\mathcal{O}(t),
\end{equation*}
где $\beta=-\frac{11}{3}\frac{C_2(G)}{(4\pi)^2}g^2$~--- бета--функция теории ($C_2(G)$~--- нормировка оператора Казимира в присоединённом представлении). Интеграл по собственному времени регуляризуем так: $\int\limits_0^\infty\frac{\D t}{t}=2\ln{\frac{\Lambda}{m}}$, где $\Lambda$~--- импульс обрезания, $m$~--- тот самый масштабный параметр теории. Всё это приведёт к следующему виду эффективного действия (в ультрафиолетовом пределе):
\begin{equation}
	\mathcal{S}^\text{eff}=-\frac{1}{4g^2}\left(1+\beta\ln{\frac{\Lambda}{m}}\right)\!\int\!\D^dx\, \tr\,F^2_{\mu \nu}.
\end{equation}

Однако, разложение по степеням $t$ непригодно для описания инфракрасного режима $t\to\infty$, так как каждый член такого ряда по отдельности будет расходиться. Поэтому в работе \cite{Mnev} было предложено представление теплового ядра в виде интеграла по траекториям. Мы вкратце изложим этот метод в главе \ref{ch:HeatKernel}. С его помощью было построено разложение эффективного действия ($\ref{eq:EffAction1}$) по производным тензора напряженности $F_{\mu \nu}$ для квантовой электродинамики.  В этой работе мы продолжаем эту идею. В главе \ref{ch:DerivExpan} мы покажем способ вычисления $\mathcal{S}^\text{eff}$ для специального случая ковариантно--постоянного поля, а также предложим метод построения  градиентного разложения в случае неабелевой калибровочной группы $SU(2)$. В дальнейшем с помощью этих результатов можно будет рассматривать более сложные и неоднородные полевые конфигурации для нахождения вакуумных возбуждений глюонов для построения теории конфайнмента.