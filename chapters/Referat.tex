% !TEX root = ../main.tex
%Дипломный проект изложен на~\formbytotal{page}{страниц}{е}{ах}{ах}, содержит \formbytotal{tablecnt}{таблиц}{у}{ы}{}, \formbytotal{figurecnt}{рисун}{ок}{ка}{ков}, \formbytotal{citeitem}{источник}{}{а}{ов}.

Целью данной работы является вычисление однопетлевых поправок к эффективному действию в теории Янга--Миллса.

В результате предложен метод расчёта однопетлевого эффективного действия и приведён пример его использования.
	
Работа состоит из двух глав, введения, заключения и трёх приложений.
Во введении в краткой форме излагаются основы теории Янга--Миллса, обсуждается актуальность проблемы.
В главе \ref{ch:HeatKernel} строится представление для однопетлевого эффективного действия в виде интеграла по траекториям.
В главе \ref{ch:DerivExpan} приводится пример применения полученного представления для случая медленно меняющегося калибровочного поля.
В заключении подводятся итоги и описываются возможные направления дальнейших работ.
В приложении \ref{sec:Calculation} вычисляется интеграл Дьяконова--Петрова по траекториям на $SU(2)$ для постоянного поля с внешним источником, необходимый для построения теории возмущений.
В приложении \ref{sec:FockSchwinger} доказывается связь калибровочного потенциала и тензора напряженности в калибровке Фока--Швингера.
В приложении \ref{sec:DPFormula} приводится явное вычисление аргумента экспоненты в формуле Дьяконова--Петрова при параметризации элемента группы $SU(2)$ углами Эйлера.